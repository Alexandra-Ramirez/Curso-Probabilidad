\documentclass[a4paper,11pt]{report}
\usepackage[sc]{mathpazo}
%\usepackage[light,firsttwo,outline,bottomafter]{draftcopy}
%\usepackage{srcltx}
\usepackage{anysize} % Soporte para el comando \marginsize
\marginsize{1.2cm}{1.2cm}{1cm}{1cm}
\usepackage{amsfonts}
\usepackage{amssymb}
\usepackage[latin1]{inputenc}
\usepackage[english,spanish]{babel}
\usepackage{amsmath}
\usepackage{multicol} 
\columnsep=7mm
\usepackage{latexsym}
\usepackage{mathrsfs}
\usepackage{indentfirst}
\usepackage{graphicx}
\usepackage{enumitem}
%\usepackage{hyperref}

\usepackage{times}



%\usepackage{xcolor}
%\usepackage{pgfplots}
%\usepackage{tikz}

% Define bar chart colors
%
%\definecolor{bblue}{HTML}{4F81BD}
%\definecolor{rred}{HTML}{C0504D}
%\definecolor{ggreen}{HTML}{9BBB59}
%\definecolor{ppurple}{HTML}{9F4C7C}





\setlength{\paperwidth}{216mm} \setlength{\paperheight}{230mm}
\setlength{\textwidth}{39pc} \setlength{\textheight}{57.5pc}
\setlength{\topmargin}{-1.5cm} \setlength{\oddsidemargin}{-0.5cm}
\setlength{\evensidemargin}{0.9cm}
%\setlength{\footskip}{-0.3cm}

\newcommand{\ds}{\displaystyle}
\newcommand{\normal}{\triangleleft \,}
\newcommand{\tx}{\textrm}
% \linespread{1.2} \sloppy


\newcommand{\Z}{\mathbb{Z}}
\newcommand{\N}{\mathbb{N}}
\newcommand{\R}{\mathbb{R}}
\newcommand{\PR}{\mathbb{P}}
\newcommand{\e}{\rightarrow}
\newcommand{\bi}{\Leftrightarrow}
\newcommand{\com}{\mathbb{N} \bi}
\newcommand{\fu}{f:\N \e \R}
\newcommand{\ba}{\backslash}
\newcommand{\Q}{\mathbb{Q}}


\newcommand{\calP}{\mathcal{P}}
\newcommand{\calF}{\mathcal{F}}
\newcommand{\calL}{\mathcal{L}}


\newcommand{\ovl}{\overline}
\newcommand{\ora}{\overrightarrow}
\newcommand{\ola}{\overleftarrow}
\newcommand{\olra}{\overleftrightarrow}
\newcommand{\ula}{\underleftarrow}
\newcommand{\ura}{\underrightarrow}


\newcommand{\inner}[2]{\langle{#1},{#2}\rangle}

%%%%%%%%%%%%%%%%%%%%%%%%%%%%%%%%

\newcommand{\cabecera}[1]{\begin{figure}[h]
 \begin{minipage}[c]{0.05\columnwidth}
\centering\includegraphics[width=2cm]{escudo.pdf}% tb escudouni.bmp
\end{minipage}
\hfill{}
\begin{minipage}[c]{0.86\columnwidth}
\centering\flushleft {Universidad Nacional de Ingenier\'ia\\
Facultad de Ciencias\\
Escuela Profesional de Matem\'atica \hfill #1}
\end{minipage}
\end{figure}\vspace{-0.5cm}
}


\begin{document}
\cabecera{Ciclo 2016-I}
\begin{center}
{\bf Examen Parcial de Introducci\'on a la Estad\'istica y Probabilidades-CM 274
}
\end{center}

\setlength{\unitlength}{1in}

\begin{picture}(6,.1) 
\put(0,0) {\line(1,0){6.25}}         
\end{picture}

 

\renewcommand{\arraystretch}{2}

\vskip.25in

\noindent {\Large \textbf{Problemas}} 

\vskip.25in

\begin{enumerate}
\item (3 ptos.) Responde y resuelve  las siguientes preguntas

\begin{enumerate}
	\item ?` Qu\'e es un camino aleatorio?
	
	 Si $S_0, S_1, S_n$ es un camino aleatorio sobre los enteros en el cual $p (= 1-q)$ es la probabilidad que un paso es dado, encuentra  que para $S_0 = 0$ un punto de tiempo $n$ y una localizaci\'on $k$,  la probabilidad de que  $S_n = k$, cuando $n $ es impar. Analiza para el caso en que $n$ es par.
	 \item Si una moneda es lanzada $n$ veces con una probabilidad de salir cara $P$ y escribimos $Y$ como el n\'umero de caras obtenidas, entonces , ?`cu\'al es el \texttt{pmf} de Y?.  
	 \item Un punto $A$ se elige en el disco unitario $\{ (x,y): x^2 + y^2 \leq 1\}$. Sea $R$ la distancia desde el origen a $A$. Encuentra $f_{R}$ y explica el sentido de la respuesta.
\end{enumerate}
\item (3 ptos)
\begin{enumerate}
	\item Muestra que, si $X$ y $Y$ son variables aleatorias  entonces  $X + Y$, $XY$ y $\min\{X,Y\} $, son tambi\'en variables aleatorias.
	\item Prueba que el conjunto de todas las variables aleatorias  forman un espacio vectorial sobre los reales. Si el espacio muestral es finito, escribe una base  para este espacio.
\end{enumerate}
\item(2ptos) Un nuevo programa de computador consiste de dos m\'odulos. El primer m\'odulo contiene un error con una probabilidad de $0.3$. El segundo m\'odulo, tiene una probabilidad de $0.35$ de tener un error independientemente del primer m\'odulo. Un error en el primer m\'odulo  causa al programa fallar con probabilidad $0.6$. Para el segundo m\'odulo, esta probabilidad es $0.75$. Si hay errores en ambos m\'odulos, el programa falla con probabilidad de $0.9$. Suponiendo que el programa ha fallado, determina la probabilidad de errores en ambos m\'odulos.

\item (4ptos) Resuelve
\begin{enumerate}
	\item  Ocho peones se colocan aleatoriamente en un tablero de ajedrez (no m\'as de uno en un cuadrado del tablero de ajedrez). ?` Cu\'al es la probabilidad que
	\begin{enumerate}
		\item ellos est\'en en  l\'inea recta (cuenta  las diagonales).
		\item no hay dos en la misma columna o fila.
	\end{enumerate}
	\item Sean $X$ y $Y$ variables aleatorias independientes con varianzas finitas y sea $U = X + Y$ y $V = XY$. Bajo que condiciones $U$ y $V$ son no correlacionadas.
	
	\item Sea $X_1, X_2, \dots X_n$ variables aleatorias independientes y supongase que $X_k$ es una variable aleatoria Bernoulli con par\'ametro $p_k$. Muestra que $Y = X_1  +X_2 + \cdots + X_n$ cuya esperanza y varianza es dada por
	
	\[
	E(Y) = \sum_{1}^{n}p_k \qquad var(Y) = \sum_{1}^{n}p_k(1 - p_k).
	\]
	\item Sea $X$ una variable aleatoria continua no negativa, con una funci\'on densidad $f$. Muestra que
	
	\[
	E(X^r) = \int_{0}^{\infty}rx^{r-1}\PR(X > x)dx \qquad \mbox{para } \ \ r\geq 1.
	\]
\end{enumerate}

\item (3 ptos) 

\begin{enumerate}
\item Encuentra la funci\'on densidad de $Y = aX$ con $a > 0$ en t\'erminos de la funci\'on densidad de $X$. Muestra que las variables aleatorias continuas $X$ y $-X$ tienen la misma funci\'on de distribuci\'on si y s\'olo si $f_{X}(x) = f_{-X}(x)$ para todo $x \in \mathbb{R}$.
\item Se traza una l\'inea por el punto (1, 0) en una direcci\'on elegida al azar. Sea $(0,Y)$ un punto  que corta el eje $Y$. Muestra que $Y$  tiene la densidad est\'andar de Cauchy:

\[
f_{Y}(y) = \dfrac{1}{\pi(1 + y^2)} \ \ (y \in \mathbb{R}).
\]

Prueba que $1/Y$ tiene la misma distribuci\'on  de $Y$. Explica geom\'etricamente. (En este ejercicio se puede ignorar la posibilidad que la l\'inea sea paralela a uno de los ejes, ya que la probabilidad ser\'ia $0$).


\end{enumerate}
\item (3ptos) Calcula 

\begin{enumerate}
	
\item El \texttt{cdf} de una variable aleatoria discreta $N$ es dado por
\[
F_{N}(x) = 1 - \dfrac{1}{2^n} \ \ \text{si }\ \ x\in[n, n +1)\ \ \text{para} \ \ n =1,2,\dots
\]	

y tiene el valor de $0$ si $x <1$.

\begin{itemize}
	\item ?` Cu\'al es el \texttt{pdf} de $N$?.
	\item Calcula $P(4 \leq N \leq 10)$.
\end{itemize}
\item Sea \texttt{cdf} de una variable aleatoria continua $X$ dada por

\[
F_{X} (x) = \begin{cases}
0 \ \ \ \ \ \ \ \ \ \ \ \ \ x <  0\\
x^2/4 \ \ \ \ \ \  0 \leq x \leq 2 \\
1 \ \ \ \ \ \ \ \ \ \ \ \  x > 2
\end{cases}
\]

Encuentra el \texttt{pdf} de $X$.
\item Sea $X$ una variable aleatoria continua cuyo \texttt{pdf} es dado por:

\[
f_{X}(x) = \begin{cases}
\alpha (1 + x)^{-3}\ \  x> 0\\
0 \ \ \ \ \ \ \ \ \ \ \ \ \text{en otros casos}
\end{cases}
\]


encuentra la funci\'on de densidad condicional $f_{X|0.25 \leq X \leq 0.5}(x)$. Comprueba tu respuesta integrando $f_{X|0.25 \leq X \leq 0.5}(x)$ entre $0.25$ y $0.5$ y mostrando que sale $1$.

\end{enumerate}

\item (2ptos)  Resuelve

\begin{enumerate}
\item La probabilidad de que un evento ocurra en un experimento es $0.5$. Usa la desigualdad de Chebyshev para mostrar que la probabilidad de que este evento ocurra entre $450$ y $550$  en $1000$ experimentos indepedientes excede de $0.9$. 
\item Una moneda muestra  una cara  cada cinco intentos, es lanzada  $100$ veces. Sea $X$ la variable aleatoria que denota el n\'umero de caras  obtenidas. Demuestra  que $E(X) = 20$ y que $Var(X)$ = 16. Ahora encuentra una cota  superior de la probabilidad  que $X \geq 60$, usando la desigualdad de Markov y la desigualdad de Chebyshev.
\end{enumerate}
\end{enumerate}
\end{document}