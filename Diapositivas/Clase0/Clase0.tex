\documentclass{beamer}
\usetheme{metropolis}  
\usecolortheme{beaver}
\usepackage{hyperref}
\usepackage{bigints}
\title{Introducci\'on a la probabilidad y estad\'istica CM274}
 \usepackage[spanish]{babel}
\date{\today}
\author{C\'esar Lara Avila}
\institute{\url{https://github.com/C-Lara}}
\begin{document}
  \maketitle
  \section{0. Acerca del curso  }
  
  
 \begin{frame}{Descripci\'on del curso}
CM274 es un curso introductorio   a las ideas fundamentales y t\'ecnicas de teoria de probabilidades e inferencia estad\'istica. Los t\'picos que incluyen son: \underline{combinatoria b\'asica}, \underline{variables aleatorias}, \underline{distribuciones de probabilidad}, \underline{teorema del l\'imites de probabilidad}, \underline{regresi\'on lineal}, etc.

\vspace{0.2cm}

El curso tiene una p\'agina web: \textcolor{blue}{\url{http://c-lara.github.io/Curso-Probabilidad/}}.
 \end{frame}

 \begin{frame}{Prerequisitos}
Los prerequisitos para este curso son:

\begin{enumerate}
 
 \item C\'alculo diferencial, integral y de varias variables.
 
 \item \'Algebra lineal.
 
 \item  Alguna familiariedad con un lenguaje de programaci\'on como Mathematica, Matlab, NumPy, C, R es asumida.
 
\end{enumerate}
 \vspace{0.5cm}
 
 Algunos enlaces importantes son:
 
 \begin{itemize}
 
\item \scriptsize{ \textcolor{brown}{\href{https://jeremykun.com/2011/07/09/set-theory-a-primer/}{Set theory a primer.}}}
 
\item \scriptsize{ \textcolor{blue}{\href{https://jeremykun.com/2011/06/19/linear-algebra-a-primer/}{Linear algebra a primer}.}}
 
\item \scriptsize{ \textcolor{orange}{\href{https://www.countbayesie.com/blog/2015/8/3/the-riemann-integral}{The Riemman integral}.}}
    
\item \scriptsize{ \textcolor{red}{\href{http://theanalysisofdata.com/probability/F_6.html}{Multivariate Differentiation and Integration}.}}
\end{itemize}
 \end{frame}
 \begin{frame}{Curso R}

 En este curso se utilizar\'a el lenguaje de programaci\'on  R (la alternativa de c\'odigo abierto al lenguaje  S), para explicar conceptos importantes de probabilidad, inferencia estad\'istica, simulaci\'on y aprovechar las t\'ecnicas novedosas de R, con respecto a otros lenguajes. 
 
 Tal vez este \textcolor{red}{\href{https://www.youtube.com/watch?v=Dx4IFguczgI}{video}} que te muestre las ventajas y desventajas del lenguaje, te animen a aprenderlo. El curso complementario del curso, acerca de R, tiene una p\'agina web:
 
  \textcolor{blue}{\url{http://c-lara.github.io/Curso-R/}}.
 

\end{frame}
\end{document}